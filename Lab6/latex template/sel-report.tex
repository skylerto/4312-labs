% !TEX encoding = UTF-8
%Koma article
\documentclass[fontsize=12pt,paper=letter,twoside]{scrartcl}
\usepackage{multirow}

%Useful preamble definitions
\usepackage[top=4cm,bottom=4cm,left=3cm,right=3cm,asymmetric]{geometry}
%\geometry{landscape}                % Activate for for rotated page geometry
%\usepackage[parfill]{parskip}    % Begin paragraphs with an empty line rather than an indent
\usepackage{graphicx}
\usepackage{amssymb}
\usepackage{epstopdf}
\DeclareGraphicsRule{.tif}{png}{.png}{`convert #1 `dirname #1`/`basename #1 .tif`.png}
% Listings needs package courier
\usepackage{listings} % Needs 
\usepackage{courier}

\usepackage[framemethod=TikZ]{mdframed}
\usepackage{url}

\usepackage{sty/bsymb} %% Event-B symbols
\usepackage{sty/eventB} %% REQ and ENV
\usepackage{sty/calculation}

%Maths
\usepackage{amssymb,amsmath}
\def\Fl{\mathbb{F}}
\def\Rl{\mathbb{R}}
\def\Nl{\mathbb{N}}
\def\Bl{\mathbb{B}}
\def\St{\mathbb{S}}
\newcommand{\ovr}{\upharpoonright}
\newcommand{\var}[1]{\textit{#1}}

\usepackage[headsepline]{scrpage2}
\pagestyle{scrheadings}
\ihead[]{\small EECS4312 Report1}
\ohead[]{\small \thepage}
\cfoot[]{}
\ofoot[]{}


%%%%PVS environment%%%%%%%%%%%%%%%%%%%
\lstnewenvironment{pvs}[1][]
    {\lstset{#1,captionpos=b,language=pvs,
    mathescape=true,
    basicstyle=\small\ttfamily,
    numbers=none,
    frame=single,
    % numberstyle=\tiny\color{gray},
    % backgroundcolor=\color{lightgray},
    firstnumber=auto
    }}
    {}
 %%%%%%%%%%%%%%%%%%%%%%%%%%%%%%%%
 
%%%%Verbatim environment%%%%%%%%%%%%%%%%%%%
\lstnewenvironment{code}[1][]
    {\lstset{#1,captionpos=b,
    mathescape=true,
    basicstyle=\small\ttfamily,
    numbers=none,
    frame=single,
    % numberstyle=\tiny\color{gray},
    % backgroundcolor=\color{lightgray},
    firstnumber=auto
    }}
    {}

% \newenvironment{boxed}[1]
%    {\begin{center}
%    #1\\[1ex]
%    \begin{tabular}{|p{0.9\textwidth}|}
%    \hline\\
%    }
%    { 
%    \\\\\hline
%    \end{tabular} 
%    \end{center}
%    }
 %%%%%%%%%%%%%%%%%%%%%%%%%%%%%%%%
 
 %Text in a box
\newenvironment{textbox}
    {\begin{center}
    \begin{tabular}{|p{0.9\textwidth}|}
    \hline\\
    }
    { 
    \\\\\hline
    \end{tabular} 
    \end{center}
    }

\title{EECS4312 Lab 6}

%%%%%%%%%%%%Enter your names here%%%%%%%%
\author{Skyler Layne (cse23170\@cse.yorku.ca)}
%%%%%%%%%%%%%%%%%%%%%%%%%%%%%%%%

\date{\today} % Display a given date or no date


\begin{document}
\maketitle

\section*{Revisions}

%%%%%%%%%%%%Table of revisions%%%%%%%%
\begin{tabular}{|l|l|p{3in}|}
\hline
Date & Revision& Description \\ 
\hline
4 November  2015
& 1.0       
& Initial specification of system, R/E- descriptions, and function table.\\ 
\hline
5 November  2015
& 2.0       
& Fixed function table to include bounds on 'i'. \\ 
\hline
\end{tabular}
%%%%%%%%%%%%%%%%%%%%%%%%%%%%%%%%
\newpage
\tableofcontents
\newpage

%%%%Rest of your document goes here%%%%%%%%%%%%%%%%%%%

%% Deactivate this to place your own work
%\input{sty/sample.tex}

%% Provide an introduction into the system.
\section{System Introduction}
\begin{figure}
\center
\includegraphics[width=.7\textwidth]{system.png}
\caption{Reference design.}
\end{figure}
{The Entry/Exit of the parking lot is a single lane passage. By controlling the indicators, the program ensures that only one car can pass through the Entry/Exit so as to prevent a car accident between entering and leaving cars.}\\

%%
\newpage
\section{E/R Descriptions}
\subsection{Requirement Descriptions}
%% R Description
\begin{tabular}{ | p{2cm} | p{6cm}  | p{6cm} |}
\hline
REQ1 & The controller will allow only one car within the passage at a time. & By monitoring sensors the system will ensure both gates will not be open at the same time. \\
\hline
\end{tabular}
\\
\\
\\
%% R Description
\begin{tabular}{ | p{2cm} | p{6cm}  | p{6cm} |}
\hline
REQ2 & A car wishing to park will have priority over the cars which want to leave.   & If both sensors are on, the entering gate will open. \\
\hline
\end{tabular}
\\
\\
\\
\subsection{Environment Descriptions}
%% E Description
\begin{tabular}{ | p{2cm} | p{6cm}  | p{6cm} |}
\hline
ENV1 & The sensors can only ever read ON or OFF.  & Only 2 states for each sensor. \\
\hline
\end{tabular}
\\
\\
\\
%% E Description
\begin{tabular}{ | p{2cm} | p{6cm}  | p{6cm} |}
\hline
ENV2 & The actuators can only ever be GO or STOP.  & Can only ever indicate either 2 states. \\
\hline
\end{tabular}
\\
\\
\\
%% List of Variables Section
\newpage
\section{List of Variables}

\subsection{Controlled Variables}

{Each of the controlled variables represents an ACTUATOR which has two states; either STOP or GO.\\} 

\begin{tabular}{ | p{2cm} | p{12cm}  |}
\hline
Name & Description  \\
\hline
X1 & Car entering sensor. ON when a car passes through the sense.    \\
\hline
X2 & Car leaving sensor. ON when a car passes through the sense.    \\
\hline
\end{tabular}
\\
\\

\subsection{Monitored Variables}
{Each of the monitored variables represents an SENSOR which has two states; either ON or OFF.\\}\\

\begin{tabular}{ | p{2cm} | p{12cm}  |}
\hline
Name & Description  \\
\hline
Y1 & Car entering indicator, indicates GO when a car would like to enter.    \\
\hline
Y2 & Car leaving indicator, indicates GO when a car would like to leave.    \\
\hline
\end{tabular}

% Function Table
\newpage
\section{Car Function Table}
{The function table presents a variable 'i' in DTIME which captures the value of both, the sensors and actuators, at each DTIME step.}
\\
\\
\begin{tabular}{|l|l|l|l|l|}
\hline
\multicolumn{3}{|l|}{}                                                                                                                                              & Y1           & Y2           \\ \hline
\multicolumn{3}{|l|}{i = 0}                                                                                                                                         & y1(i) = go   & y2(i) = stop \\ \hline
\multirow{6}{*}{i \textgreater 0} & \multicolumn{2}{l|}{x1(i) = on AND x2(i) = on}                                                                                  & y1(i) = go   & y2(i) = stop \\ \cline{2-5} 
                                  & \multirow{2}{*}{x1(i) = off AND x2(i) = on} & \begin{tabular}[c]{@{}l@{}}y1(i-1) = stop AND y2(i-1)\\ = stop\end{tabular}       & y1(i) = stop & y2(i) = go   \\ \cline{3-5} 
                                  &                                             & \begin{tabular}[c]{@{}l@{}}NOT (y1(i-1) = stop AND\\ y2(i-1) = stop)\end{tabular} & y1(i) = stop & y2(i) = stop \\ \cline{2-5} 
                                  & \multirow{2}{*}{}                           & \begin{tabular}[c]{@{}l@{}}y1(i-1) = stop AND y2(i-1)\\ = stop\end{tabular}       & y1(i) = go   & y2(i) = stop \\ \cline{3-5} 
                                  &                                             & \begin{tabular}[c]{@{}l@{}}NOT (y1(i-1) = stop AND\\ y2(i-1) = stop)\end{tabular} & y1(i) = stop & y2(i) = stop \\ \cline{2-5} 
                                  & \multicolumn{2}{l|}{x1(i) = off AND x2(i) = off}                                                                                & y1(i) = go   & y2(i) = stop \\ \hline
\end{tabular}
%\end{table}


%%%%%%%%%%%%%%%%%%%%%%%%%%%%%%%%%%%%%

\end{document}  